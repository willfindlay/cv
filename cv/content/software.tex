%-------------------------------------------------------------------------------
%	SECTION TITLE
%-------------------------------------------------------------------------------
\cvsection{Open-Source Software}

%-------------------------------------------------------------------------------
%	SUBSECTION TITLE
%-------------------------------------------------------------------------------
\cvsubsection{Creator/Maintainer}


%-------------------------------------------------------------------------------
%	CONTENT
%-------------------------------------------------------------------------------
\begin{cventries}

%---------------------------------------------------------
  \cventry
    {eBPF-Based Process Confinement Mechanism} % Quick description
    {bpfbox} % Title
    {} % Empty
    {} % Empty
    {
      \begin{cvitems} % Description(s)
        \item Designed and implemented the first eBPF-based policy enforcement engine and a high-level policy language for process confinement.
        \item This work was \href{https://dl.acm.org/doi/10.1145/3411495.3421358}{published at ACM CCSW'2020}.
        \item Full source code available: \url{https://github.com/willfindlay/bpfbox}
      \end{cvitems}
    }

%---------------------------------------------------------
  \cventry
    {eBPF-Based Intrusion Detection System} % Quick description
    {ebpH} % Title
    {} % Empty
    {} % Empty
    {
      \begin{cvitems} % Description(s)
        \item Designed and implemented an intrusion detection system for Linux based on eBPF.
        \item Establishes per-executable system call profiles in order to establish normal behaviour and detect anomalies.
        \item Full source code is available: \url{https://github.com/willfindlay/ebpH}.
      \end{cvitems}
    }

%---------------------------------------------------------
  \cventry
    {Experimental libbpf Bindings for Python} % Quick description
    {pybpf} % Title
    {} % Empty
    {} % Empty
    {
      \begin{cvitems} % Description(s)
        \item Designed and implemented an experimental eBPF framework for Python with support for CO-RE and libbpf bindings.
        \item Full source code is available: \url{https://github.com/willfindlay/pybpf}.
      \end{cvitems}
    }

%---------------------------------------------------------
\end{cventries}

%-------------------------------------------------------------------------------
%	SUBSECTION TITLE
%-------------------------------------------------------------------------------
\cvsubsection{Contributor}


%-------------------------------------------------------------------------------
%	CONTENT
%-------------------------------------------------------------------------------
\begin{cventries}

%---------------------------------------------------------
  \cventry
    {eBPF Programming Framework for Python} % Quick description
    {bcc} % Title
    {} % Empty
    {} % Empty
    {
      \begin{cvitems} % Description(s)
        \item Regular contributor to a large open-source project.
        \item Implemented the following major features:
        \begin{itemize}
          \item \href{https://github.com/iovisor/bcc/commit/fe730f29f14bef8b5ffe1112c578df876c44d22d}
            {Support for the ringbuf eBPF map}
          \item \href{https://github.com/iovisor/bcc/commit/9b82af3ef53bbae76d9f09f403b58975995aa900}
            {Enhanced support for LSM probes}
          \item \href{https://github.com/iovisor/bcc/commit/e70bbdcbcbcd01e5570ba7b9d79e282d16a53d40}
            {Python support for stack and queue eBPF maps}
        \end{itemize}
        \item Full source code is available: \url{https://github.com/iovisor/bcc}.
      \end{cvitems}
    }

%---------------------------------------------------------
\end{cventries}

